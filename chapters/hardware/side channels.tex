\section{Side Channels}
At any given moment, in the system informations are constantly flowing, because
it is an intrinsic phenomenon of computation. This is true both for a single computer system and a
distributed one.\\
Anyway, all those information flows use intended, specific and implemented channels, in which the
data requires some security guarantees, such as integrity, availability, non-repudiation, \dots\\

\begin{section}{Covert Channels}
  \begin{boxH}
    A covert channel is a path for an illegal flow of information within a system.
  \end{boxH}
  Any communication channel that a process can exploit to transfer information in a manner that
  violates the system’s security policy.

  There are many types of covert channels within a computing system, based on what the covert channel
  actually is:
  \begin{itemize}
    \item Timing covert channels
    \item Termination covert channels
    \item Probability covert channels
    \item Resource utilization covert channels
    \item Power covert channels
  \end{itemize}

  Side channels are most flexible, because they can be used for a large variety of uses, both to
  improve or undermine the security of a system.\\
  Every side channel have some defining characteristics:
  \begin{itemize}
    \item the existent, or if the channel is a potential one
    \item the bandwidth of the channel, or how much information can be transmitted
    \item how noisy it is, because informations can be distorted or lost by it
  \end{itemize}
  It is usually infeasible for realistic systems to eliminate every potential covert
  channel, so some usable techniques against them are to inject noise in the channel or monitor the
  pattern used to exploit the channel.
\end{section}
