\chapter{Physical Unclonable Functions}
\label{sec:PUF}

\begin{boxH}
  Physical Unclonable Functions (PUFs) have been introduced as the hardware equivalent of a one-way
  function
\end{boxH}

Some examples of them are:
\begin{itemize}
  \item things that add some delay on some networks, like Ring Oscillators PUF and Arbiter PUF
  \item the content of the SRAMs at boot time
  \item the re
\end{itemize}

A PUF must be:
\begin{itemize}
  \item computable
  \item unique
  \item reproducible: $y\approx PUF(x)$
  \item unclonable 
  \item unpredictable
  \item one-way, meaning that it is hard to invert the function, even with some errors
\end{itemize}

The PUF is a obviously a function, to which a challenge is provided and a response is obtained. In
the case of a memory address as a input challenge, the response is surely unique.
We still need to get reproducible results, because with the same input we expect the same outputs,
so we expect some, small, errors, otherwise it would be a random function

There are actually different kinds of PUFs, like silicon and non-silicon PUFs. 

Ring oscillators PUF are produces a sequentiual output, with some stability in the circuit.
There are also ways to know how many bits are neeed base on the number of needed output and the
number of rings oscillators.
