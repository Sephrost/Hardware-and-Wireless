\chapter{Test-Infrastructure based attacks}
\begin{boxH}
  \textbf{Test-Infrastructure based attacks} are a kind of \textbf{active attacks}, that are
  performed on the test infrastructure of a device.
\end{boxH}
This kind of attacks can be performed just after manufacturing, and thus before entering the market,
or after the device are sold and are still unused.
Tests are mandatory on an IC to guarantee the quality of the product, but during them little of the
power and resources are used, to guaranty that the test is being performed correctly. To perform the
tests some tests facilities are introduced, but those can be exploited to performs some kinds of
attacks, because they are not removed the final chip, because they could be needed later on.
They are also kept because they can be used to test if the system has been compromised. They are
there to stay, and no access limitations are put up in place.\\

Among the plethora of test infrastructures:
\begin{itemize}
  \item Scan chains
  \item Standard IEEE 1149.1
  \item JTAG infrastructures, that includes the pins
  \item ATPG, the automatic test pattern generation, which is the physical and software facility
    that is used to generate the test patterns(input and output configurations that, if the test is
    wrong allows to retrieve where the error occured)
\end{itemize}
In scan-based devices, the scan chains could provide a natural way to access the content of ALL the
storage devices (flip-flops and registers) connected to a scan chain.\\
Thus, potential scan chains can contain a secret (directly or indirectly):
\begin{itemize}
  \item Directly: the secret key itself
  \item Indirectly: a value that is a function of the secret key (e.g., an intermediate value during
    the encryption process)
\end{itemize}
\begin{boxH}
  To sum it up, the goal of a \textbf{scan-based attack} is to retrieve embedded secret data by 
  exploit observability and controllability offered by scan chains, by basically toggling the
  circuit between functional and scan mode.
\end{boxH}
\begin{section}{Insight on teaming}
  Test-infrastructure are relevant for the different teams in cybersec.
  \begin{subsection}{Red team}
    The red team is the one that attacks the system, and tries to find vulnerabilities in the system.
    In some architectures, scan chains could:
    \begin{itemize}
      \item Be hidden
      \item Be accessible just via additional infrastructures, such as 1149.1 cells
      \item Be managed via complex Scan compression Codec
    \end{itemize}
    In this context, to attack a scan chain, the red team has to:
    \begin{enumerate}
      \item Identify the scan chain SC to which the target FFs belong
      \item Identify the precise time instant T in which the target FFs contain the secret.
      \item Run the circuit in Normal mode until the time T
      \item Stop the circuit and switch it to Test mode
      \item Scan out the content of the scan chain SC until the content of all the target FFs reach an output
        point you can observe
      \item If Test decompressors and compressors are present, you have to reverse them previously
    \end{enumerate}

  \end{subsection}

  \begin{subsection}{Blue team}
    The blue team is the one that defends the system, and tries to find the vulnerabilities before the
    To protect the system from this kind of attacks, the blue team can use some countermeasures,
    like:
    \begin{itemize}
      \item unbounding the chain, using a fuse in a way that the chain is inaccessible without the
        correct approach, because the fuse is blown and the chains became inaccessible
      \item replace scan chains with Built-In Self-Test (BIST) solutions. This allows to avoid
        scan-based testing, while still allowing at-speed testing, but at the cost of a higher area 
        overhead and a lower fault coverage and diagnostic resolution
      \item introduce a secure test access mechanism, with is expensive (authN need to be
        introduced) with no easy way to implement it
      \item scan chain encryption, which is a way to encrypt the scan chain
    \end{itemize}

  \end{subsection}
\end{section}
