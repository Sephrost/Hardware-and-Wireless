\begin{chapter}{Fault Attacks}
  Fault attacks consist in injecting deliberate (malicious) faults into the target device, aimed at
  bringing it into a set of states from which private internal information items (e.g., a key) can
  be fraudulently extracted.\\
  They idea is pretty simple: take the case of a encryption algorithm, if i run it and get a 1, by
  running that again injecting a fault, we are able to understand what the original bit was. In a
  more formal way, if $C_{OK}=E(P)$ and $C_{OK}'=E(P)$ while injecting a fault, if $C_{OK} \neq
  C_{OK}'$, then the bit is 0, otherwise it is 1.\\

  There are many ways to induce a fault in a system, some of them are:
  \begin{itemize}
    \item Underpowering or overvolting the device
    \item altering the clock
    \item changing the temperature
    \item injecting a laser in the device, depackaging it beforehand and shining the laser on the
      silicon
    \item electromagnetic injections
  \end{itemize}

  By injecting the system with faults, it is possible to achieve many interesting results, like:
  \begin{itemize}
    \item bypassing the security checks
    \item generate faulty outputs, like wrong encrypted messages
    \item generate denial of service
  \end{itemize}

  A set of multiple injections with subsequent correlated analysis of the faulty behavior of the
  target device
  \begin{itemize}
    \item The attackers (cryptanalysts) focus on the fault influences of specific hardware
      implementations to collect some faulty outputs.
    \item Later, this information is compared and analyzed with correct results to harvest some
      partial or total compromise of the secret information.
  \end{itemize}

  \begin{section}{Fault Injections Countermeasures}
    Several approaches have been proposed, including:
    \begin{itemize}
      \item Preventing the attack
      \item Detecting the fault injection
      \item Detecting the fault effect (error)
      \item De-synchronization
      \item Robust package, “hardened” technologies
    \end{itemize}

    \begin{subsection}{Fault Detection}
      One basic defense system is to detect when the attack is happening, by using some kind of
      sensors, and are ths connected to the root of trust. The problems is that they can generate
      false positives in very soisy environments, and even in ideal conditions, attacks are still
      pssible to some degree.
    \end{subsection}

    \begin{subsection}{Error Detection}
      Another approach is still based on detection, but we introduce redundancy in the system to
      detect when an error is happening. The redundancy can be multiple execution or inverting the
      operation, for example. Detection is not a very effective mechanism because it is not possible
      to tell which result is the faulty one, and which is the correct one.\\
      Many standards cover fault, eg. fault tollerant systems, and thus they can be used to be a
      valuable countermeasure against fault attacks.
    \end{subsection}

    \begin{subsection}{De-synchronization}

    \end{subsection}
    \begin{subsection}{Hardened IC Package}

    \end{subsection}
  \end{section}

\end{chapter}
