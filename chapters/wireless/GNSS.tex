\chapter{GNSS and positioning}
  Global Navigation Satellite Systems (GNSS) are satellite-based systems that provide positioning, 
  navigation, and timing (PNT) services to users worldwide. The most well-known GNSS is the Global 
  Positioning System (GPS).\\
  The use of GPS and other GNSSs has become quite common in several civil application fields, (e.g.,
  transports and personal mobility, agriculture, ICT, ...), so Location Based Services (LBS) are
  becoming more and more important.\\

  In this chapter we will that GNSS users are particularly sensitive to some kinds of attacks, because
  those signals are very weak: for example Replaying broadcast GNSS signals with some delay 
  (meaconing), is enough to trick a user into a wrong position.\\
  \begin{section}{The positioning Problem}
    First of all, we need to define what is a position.\\
    \begin{boxH}
      A position is a set of coordinates, one for each dimension, associated with a reference 
      system.
    \end{boxH}
    \begin{figure}[h]
      \centering
      \includegraphics[width=0.8\textwidth]{img/wireless/GNSS position.png}
      \caption{Positioning in a 3D space}
      \label{fig:GNSS position}
    \end{figure}
    As shown in picture \ref{fig:GNSS position}, ideally we would like to have a system that can
    get a precise point positioning over a space, but in reality what we get is a cloud of points
    around the real position, from which the real position is estimated, withing a certain confidence
    interval.\\

    That being said, how is a position estimated?\\
    The information needed to estimate a position is provided by a set of sensors installed on 
    a terminal, such as a smartphone or a GPS receiver.\\
    Some of the most common sensors used are:
    \begin{itemize}
      \item \textbf{Satellite Navigation Chipset}, such as a GPS sensor, which tracks the signals
        from the satellites.
      \item \textbf{Inertial systems}, such as accelerometers and gyroscopes that measure the 
        acceleration and the angular velocity of the terminal.
      \item \textbf{Electronic Compasses}, which provide the orientation of the terminal.
      \item \textbf{Barometers}, which measure the atmospheric pressure.
      \item $\dots$
    \end{itemize}

    Because using all that sensors together to estimate a position is ideal but quite complex,
    a \textbf{data fusion algorithm} is used to combine the information from the sensors and estimate
    the position.\\
    \begin{figure}[h]
      \centering
      \includegraphics[width=0.3\textwidth]{img/wireless/data fusion algorithms.png}
      \caption{Representation of a data fusion algorithm}
      \label{fig:GNSS data fusion}
    \end{figure}
    In some cases using only the information provides is not enough to estimate a position, so
    some \textit{tricks} are used to improve the accuracy of the position estimation.\\
    For example google maps, when it gets an off road position, uses the information provided and
    maps the position to a point between some landmarks positioned on the road, as shown in picture
    \ref{fig:maps example}.\\

    \begin{figure}[h]
      \centering
      \includegraphics[width=0.6\textwidth]{img/wireless/maps example.png}
      \caption{Example of position estimation using landmarks}
      \label{fig:maps example}
    \end{figure}
  \end{section}

  \begin{section}{Satellite Navigation Systems and Segments}
    A little of history: the first satellite navigation system was put in space by the US Department
    of Defense in 1978. But it was only in the early 2000s that the system was opened to civil users.\\
    The european counterpart of the GPS is the Galileo system, which was put in space in 2017.\\
    \begin{figure}[h]
      \centering
      \includegraphics[width=0.9\textwidth]{img/wireless/GNSS history.png}
      \caption{History of GNSS systems}
      \label{fig:GNSS systems history}
    \end{figure}
    To provide a global coverage, a GNSS system is composed of a constellation of satellites, which
    are positioned in such a way that at least 4 satellites are visible from any point on the Earth.\\

    There are 4 main GNSS systems nowadays:
    \begin{itemize}
      \item \textbf{GPS} (USA)
      \item \textbf{Galileo} (Europe)
      \item \textbf{GLONASS} (Russia)
      \item \textbf{Beidou} (China)
    \end{itemize}
    which orbits at different altitudes and inclinations.\\
    Each of those systems  continuously transmit navigation signals in different frequencies in $L$ 
    band, as shown in picture \ref{fig:GNSS band}.\\

    \begin{figure}[h]
      \centering
      \includegraphics[width=0.8\textwidth]{img/wireless/GNSS band.png}
      \caption{Division of the $L$ band for GNSS systems}
      \label{fig:GNSS band}
    \end{figure}

    All those system are composed of 3 main segments:
    \begin{itemize}
      \item \textbf{Space Segment}, which is composed of the satellites, which continuously transmit
        the signals.
      \item \textbf{Control Segment}, which is composed of the ground stations that monitor the 
        satellites and send them the corrections.
      \item \textbf{User Segment}, which is composed of the users that receive the signals from the
        satellites and estimate their position. It performs 3 core functions:
        \subitem - Identification of the satellites in view
        \subitem - Measurement of the user-satellite distance
        \subitem - Run a PVT(Position, Velocity, Time) estimation algorithm
    \end{itemize}
    \begin{figure}[h]
      \centering
      \includegraphics[width=0.2\textwidth]{img/wireless/GNSS segments.png}
      \caption{Schema of the GNSS segments}
      \label{fig:GNSS segments}
    \end{figure}
    \begin{subsection}{GNSS signals}
      The GNSS satellites continuously transmit navigation signals in different frequencies in $L$ 
      band.\\
      Those signals are made up of different components, as shown in picture \ref{fig:GNSS signals}.\\
      The carrier modulate the data at the right frequency, the spreading code to perform multiple
      access, allowing to distinguish information from different satellites and demultiplexing the
      signals from the same satellite, , and the navigation data that contains the information 
      about the satellite position and the time.\\

      The navigation data allow the user to compute the travelling time from the satellite to the receiver,
      and the satellite position at any time.\\
      Using both the satellite position and the travelling time, the user can estimate its position
      , velocity and time(PVT).\\

      \begin{figure}[h]
        \centering
        \includegraphics[width=0.8\textwidth]{img/wireless/GNSS signal scheme.png}
        \caption{Scheme of the signals that make up the GNSS signal}
        \label{fig:GNSS signals}
      \end{figure}
    \end{subsection}

  \end{section}
  \begin{section}{Multilateration}
    Lets take a look at figure \ref{fig:GNSS multilateration problem}.\\
    We have a satellite that sends a signal $R_i$ to a user $x$, which would like to estimate 
    three coordinates $\langle x, y, z \rangle$.\\
    The signal contains the position of the satellite and the departure time of the message
    $T_{TX}$.\\
    The receiver measures the arrival time of the signal $T_{RX}$, and computes the distance
    between the receiver and the satellite $R_i$ as follows:
    \begin{equation}
      R_i = c\cdot \tau = c(T_{RX} - T_{TX})
      \label{eq:GNSS distance}
    \end{equation}
    where $c$ is the speed of light, which is the propagation speed of the signal, and $\tau$ is the
    \textit{time of flight} of the signal.\\
    In other words, the distance is the traveling time of the signal multiplied by propagation time
    in the free space.\\

    \begin{figure}[h]
      \centering
      \includegraphics[width=0.8\textwidth]{img/wireless/GNSS multilateration problem.png}
      \caption{Multilateration problem}
      \label{fig:GNSS multilateration problem}
    \end{figure}

    One measurement is not enough to estimate the position of the receiver, so at least 3 measurements
    are needed.\\
    \begin{boxH}
      We call this procedure \textbf{multilateration}, or the process of determining the position of an object
      by measuring the time of arrival of the signals from multiple sources.
    \end{boxH}

    But knowing the distance only tells us that the receiver is on a sphere centered on the satellite
    with radius $R_i$. It can be at any point on the sphere.\\
    If we have 2 satellites, we have 2 spheres, and the receiver is at the intersection of the two
    spheres.\\
    If we have 3 satellites, we have 3 spheres, and the receiver is at the intersection of the 3
    spheres.\\
    Unfortunately things are never that simple, but it gives the rough idea of how multilateration
    works.\\

    \begin{figure}[h]
      \centering
      \includegraphics[width=\textwidth]{img/wireless/multilateration comparison.png}
      \caption{Multilateration with 1, 2 and 3 satellites}
      \label{fig:Multilateration comparison}
    \end{figure}
  \end{section}
  \begin{section}{GNSS Functional basics}
    Lets add more detail to multilateration.\\
    If a satellite trasmit a pulse at time $t_0$, the receiver will receive it at time $t_0+\tau$,
    where $\tau$ is the time it takes for the signal to travel from the satellite to the receiver.\\
    We already saw in equation \ref{eq:GNSS distance} that the distance between the satellite and the
    receiver is equal to the time of flight of the signal multiplied by the speed of light.\\

    To get the distance, we only need to timestamp correctly the time of arrival of the signal.
    Yet, calculating the time of flight is not that simple, because the satellite and the receiver
    must be \textbf{synchronized} to allow the transmission time and the reception time to be on the 
    same time scale, and to be precise. Otherwise the distance estimation would be very off: 
    a signal that travels at the speed of light for 1 second travels 300000 km.\\
    In fact, the satellite host an \textbf{atomic clock}, which is a very precise clock that allows
    the satellite to timestamp the signal with a very high precision. They are also synchronized thanks
    to the control segment.\\
    The receiver also has a clock, but it is most likely not as precise as the atomic clock. 
    For this reason, the bias of the user torward the GNSS time scale $\delta t_u$ is unknown.\\
    The measured distance is then different from the geometric range and it is referred
    to as \textbf{pseudorange}, which is defined as:
    \begin{equation}
      \rho = c \cdot \tau + c \cdot \delta t_u
      \label{eq:GNSS pseudorange}
    \end{equation}

    There's also the bias of the satellite toward the scale $\delta t_u$. 
    This value is usually small and stable over time. The ground segment computes a bias $\delta t^S$
    that is uploaded to the satellite, and that the receiver can use to compute the right time of 
    flight.\\
    For those reason, we consider it to be zero, because it can be corrected.\\
    On the other hand, $\delta t_u$ cannot, the only thing we can do is keep it, so the position 
    is estimated by measuring the different pseudoranges from 4 different satellites, the minimum
    number of satellites in Line of Sight to estimate the position.\\
    f a larger number of satellites is in view a better estimation is possible.\\

    \begin{figure}[h]
      \centering
      \includegraphics[width=0.8\textwidth]{img/wireless/GNSS bias.png}
      \caption{Graphical representation of the bias $\delta t_u$ on the GNSS timescale}
      \label{fig:GNSS pseudorange}
    \end{figure}

    To sum it up, in GNSSs the user position is obtained through:
    \begin{itemize}
      \item \textbf{satellites} that broadcast that transmits timestamps.
      \item \textbf{Pseudoranges}, estimated trough one-way-arrival measurements
        receiver, plus the bias of the user toward the GNSS time scale.
      \item \textbf{Multilateration} based on the measurements
    \end{itemize}

    \begin{subsection}{The estimation problem}
      We have the measurements, but we need to estimate the position of the receiver.\\
      The relationship between the pseudoranges can be written as:
      \begin{equation}
        \begin{cases}
          \rho_1 = \sqrt{(x_1 - x_u)^2 + (y_1 - y_u)^2 + (z_1 - z_u)^2} + c\delta t_u\\
          \rho_2 = \sqrt{(x_2 - x_u)^2 + (y_2 - y_u)^2 + (z_2 - z_u)^2} + c\delta t_u\\
          \rho_3 = \sqrt{(x_3 - x_u)^2 + (y_3 - y_u)^2 + (z_3 - z_u)^2} + c\delta t_u\\
          \rho_4 = \sqrt{(x_4 - x_u)^2 + (y_4 - y_u)^2 + (z_4 - z_u)^2} + c\delta t_u\\
        \end{cases}
        \label{eq:GNSS pseudorange relationship}
      \end{equation}
      where $\langle x_i, y_i, z_i \rangle$ are the coordinates of the satellites(which are known),
      and $\langle x_u, y_u, z_u \rangle$ are the coordinates of the user(which are unknown as the bias).

      The goal is to invert the relationship and find the user coordinates and the bias given the
      pseudoranges.\\
      It is a non-linear estimation problem which is linearized through Taylor expansion 
      And solved through Least Mean Square Solutions or Bayesian Filters (e.g. Kalman).\\

      The generic pseudorange, which is a nonlinear relationship,
      \begin{equation*}
        \rho_i = \sqrt{(x_i - x_u)^2 + (y_i - y_u)^2 + (z_i - z_u)^2} + c\delta t_u
      \end{equation*}
      can be approximated through the Taylor expansion around a known location used as a linearization point
      (a point that is close to the real solution) $\langle \hat{x}_u, \hat{y}_u, \hat{z}_u, \hat{\delta t}_u \rangle$:
      \begin{equation*}
        \hat{\rho}_i = \sqrt{(x_i - \hat{x}_u)^2 + (y_i - \hat{y}_u)^2 + (z_i - \hat{z}_u)^2} + c\hat{\delta t}_u
        \label{eq:GNSS linearization}
      \end{equation*}
      To put it into a graphical prospective, take a look at figure \ref{fig:GNSS linearization}.\\
      If we choose a linearization point, we need to know the difference $\Delta x_u$ between the real
      position and the linearization point. If we can estimate it, which is easier because it is a 
      linear problem, we can estimate the user position.\\


      \begin{figure}[h]
        \centering
        \includegraphics[width=0.5\textwidth]{img/wireless/linearization scenario.png}
        \caption{Graphical representation of the linearization of the pseudorange in one dimension}
        \label{fig:GNSS linearization}
      \end{figure}
      We can expand this concept to the general formula \ref{eq:GNSS linearization} as the difference 
      between the real position and the linearization point:
      \begin{equation}
        \begin{cases}
          \Delta x_u = x_u - \hat{x}_u\\
          \Delta y_u = y_u - \hat{y}_u\\
          \Delta z_u = z_u - \hat{z}_u\\
          \Delta \delta t_u = \delta t_u - \hat{\delta t}_u\\
        \end{cases}
        \label{eq:GNSS linearization difference}
      \end{equation}
      The linearized pseudorange can be written as:
      \begin{equation}
        \delta \rho_j = \hat{\rho}_j - \rho_j = a_{xj}\Delta x_u + a_{yj}\Delta y_u + a_{zj}\Delta z_u + a_{tj}\Delta \delta t_u
        \label{eq:GNSS linearized pseudorange}
      \end{equation}
      The coefficients $a_{ij}$ are the partial derivatives of the pseudorange with respect to the
      user position and the bias, obtained through the Taylor expansion.\\
      They can be written as:
      \begin{equation}
        \begin{cases}
          a_{xj} = \frac{x_j - \hat{x}_u}{\hat{\rho}_j}\\
          a_{yj} = \frac{y_j - \hat{y}_u}{\hat{\rho}_j}\\
          a_{zj} = \frac{z_j - \hat{z}_u}{\hat{\rho}_j}\\
          a_{tj} = \frac{c}{\hat{\rho}_j}
        \end{cases}
        \label{eq:GNSS linearized coefficients}
      \end{equation}
      where 
      \begin{equation*}
        \hat{\rho}_j = \sqrt{(x_j - \hat{x}_u)^2 + (y_j - \hat{y}_u)^2 + (z_j - \hat{z}_u)^2} + c\hat{\delta t}_u
      \end{equation*}
      that being the geometrical distance between the linearization point and the satellite.\\
      This allows us to write our relationship as:
      \begin{equation}
        \begin{cases}
          \delta \rho_1 = a_{x1}\Delta x_u + a_{y1}\Delta y_u + a_{z1}\Delta z_u + a_{t1}\Delta \delta t_u\\
          \delta \rho_2 = a_{x2}\Delta x_u + a_{y2}\Delta y_u + a_{z2}\Delta z_u + a_{t2}\Delta \delta t_u\\
          \delta \rho_3 = a_{x3}\Delta x_u + a_{y3}\Delta y_u + a_{z3}\Delta z_u + a_{t3}\Delta \delta t_u\\
          \delta \rho_4 = a_{x4}\Delta x_u + a_{y4}\Delta y_u + a_{z4}\Delta z_u + a_{t4}\Delta \delta t_u\\
        \end{cases}
        \end{equation}
        that can be also rewrited in a matrix form as:
        \begin{equation}
        \begin{bmatrix}
          \delta \rho_1\\
          \delta \rho_2\\
          \delta \rho_3\\
          \delta \rho_4\\
        \end{bmatrix}
        =
        \begin{bmatrix}
          a_{x1} & a_{y1} & a_{z1} & a_{t1}\\
          a_{x2} & a_{y2} & a_{z2} & a_{t2}\\
          a_{x3} & a_{y3} & a_{z3} & a_{t3}\\
          a_{x4} & a_{y4} & a_{z4} & a_{t4}\\
        \end{bmatrix}
        \begin{bmatrix}
          \Delta x_u\\
          \Delta y_u\\
          \Delta z_u\\
          \Delta \delta t_u\\
        \end{bmatrix}
        \label{eq:GNSS linearized matrix}
      \end{equation}
      where $a_{tj}$ can be approximated as $1$ because the bias is usually small.\\
      This matrix representation is easier to invert, in fact 
      \begin{equation*}
        \delta x= H^{-1}\delta \rho
      \end{equation*}
      which is our solution: the user position.\\
      \begin{subsubsection}{The Least Squares Solution}
        The method above can only be used if the number of satellites is small(4 or less), because 
        otherwise the matrix $H$ would be non-square and non-invertible.\\
        If the number of satellites is larger, it is possible to use the Least Squares Solution.\\
        The solution would be 
        \begin{equation}
          \Delta x = (H^TH)^{-1}H^T\delta \rho
          \label{eq:GNSS least squares}
        \end{equation}
        where $H^T$ is the transpose of the matrix $H$.\\
      \end{subsubsection}

      \end{subsection}

    \end{section}

