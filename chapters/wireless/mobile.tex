\chapter{Mobile Networks}
The mobile telephones were at first introduced in the 60s, but were not so popular and required an
operator to connect the calls. Ever since, the technology has swiftly evolved, and now we have
smartphones, which are able to do many other things outside voice calls, and today there are even
more mobile phones than people in the world.\\ 
All those device are connected trough Mobile Networks, which allow radio user terminals , named User
Equipment(UE) or Mobile Stations(MS), to connect to global networks infrastructure, like the
Internet. A mobile network is divided into two parts: the \textbf{Radio Access Network}(RAN), which
is The part of the network that connects the UE to the core network, and the \textbf{Core
Network}(CN) itself, which is the part of the network that connects the RAN to the global network
infrastructure.\\
\begin{figure}[h]
  \centering
  \includegraphics[width=0.5\textwidth]{img/wireless/mobile network overview.png}
  \caption{The overview of a mobile network}
  \label{fig:mobile-network}
\end{figure}
In the 2G and 3G technologies, the CN that provided access to the telephone service is based on
circuit-switched technology, which means that the connection between the two users is established
before the communication starts, and a separate core network is based on packet-switched technology,
to provide internet access. This was reverted after the 4G technology, in which the CN is based on
packet-switched technology to provide both telephone and internet services.\\
The main architectural elements of RAN are Base Transceiver Stations (BTS or BS) that
connect to UEs through a radio interface. For the newers technologies, like 4G and 5G, the BTS are
connected directly to the CN, but for the older ones, like 2G and 3G, the BTS are connected to an
additional radio controller placed between the BTS and the CN.\\
\begin{figure}[h]
  \centering
  \includegraphics[width=0.7\textwidth]{img/wireless/mobile base stations.png}
  \caption{An overview of the placement of the base stations}
  \label{fig:mobile-architecture}
\end{figure}
\begin{section}{Cellular network Cells}
  Nowadays, the BSs are placed practically everywhere, because the SNR scales quadratically with the
  distance, to provide connectivity at any position, and the UEs can connect to the most convenient
  BS, which is usually the closest one. The area covered by a BS is called a \textbf{cell}.\\
  Historically, the cells share is considered \textit{hexagonal}, but in reality they are not, because
  signals do not propagate in a straight line, and the cells are usually irregularly shaped. The
  hexagonal approximation is used to simplify the calculations.
  \begin{subsection}{Frequency reuse}
    Radio waves are a limited resource, and the same frequency cannot be exclusively dedicated to a
    channel in a cell. But if the same frequency is reused, which is the basic idea, it generate
    interference, which is not ideal, so the cells are divided into \textbf{clusters}, which allows
    to reuse frequencies provided that the interference is limited and constrained. Each cluster
    have a pattern, that can be repeated for each cluster without generating too much interference.
    After all, if we consider a individual cell to have homogeneous traffic, and $k$ frequencies
    have to be assigned between a group of cells, you can notice that this problem is reducible to a
    \textbf{graph coloring problem}, which is NP-hard.
  \end{subsection}

  \begin{subsubsection}{Cell sizes}
    There are actually different sized of cells, which are used to provide different coverage and
    capacity. 
    \begin{itemize}
      \item \textbf{Macrocells} are the largest cells, and are used to provide coverage in rural
        areas, where the density of users is low.
      \item \textbf{Microcells} are used to provide coverage in urban areas, where the density of
        users is high, and the cells are smaller to provide more capacity.
      \item \textbf{Picocells} are used to provide coverage in buildings, like offices or shopping
        malls, where the density of users is very high.
      \item \textbf{Femtocells} are used to provide coverage in homes, where the density of users
        issues very high.
    \end{itemize}
    \begin{figure}[h]
      \centering
      \includegraphics[width=0.4\textwidth]{img/wireless/cell sizes.png}
      \caption{The different sizes of cells}
      \label{fig:cell-sizes}
    \end{figure}
  \end{subsubsection}

  \begin{subsection}{Mobility Management}
    Users are not static in space, but they are able to move freely in the cell, and between cells.
    The network must thus be able to manage the mobility of the users, and to do so it uses a
    mechanism called \textbf{Handover}, tracking the user's position(in term of cell) and changing
    the network to adapt routing and commands to with the connection to the new cell.\\
    \begin{figure}[h]
      \centering
      \includegraphics[width=0.5\textwidth]{img/wireless/mobile handover.png}
      \caption{An overview of the handover process}
      \label{fig:handover}
    \end{figure}
    The network is constantly monitored to check the signal strength and the quality of each
    neighboring cell, in respect to the MS, which measurements are sent to the BS every 480ms(at
    least for GMS). When the signal strength of the current cell is drops below a certain
    \textbf{threshold}, or becomes lower than the signal strength of a neighboring cell, the
    handover process is started by sending an \textbf{handover request} to the MS via the BTS which
    is serving the MS. While the MS is preparing, the network sends an \textit{Immediate Assignment}
    message to the MS, which contains the frequency and the time slot to use to connect to the new
    cell, and will be followed by the MS. Once the MS is successfully handed over to the new cell,
    data and voice traffic are transferred to the new cell, and after it is completed by a
    \textit{Handover confirmation} message, the old connection is released.\\
    If the handover process results in a change in the Base Station Controller (BSC) or Mobile
    Switching Center (MSC), new connections are established with the new controllers.The old BSC and
    MSC release resources associated with the call and update their databases with the MS’s new
    location information.
  \end{subsection}

  \begin{subsection}{Mobile Updates}
    The users can still move but not want an handover to be performed, for example when the user is
    moving very fast. In this case, the network is then divided in Location Areas(LAs), which are
    a set of cells. The MS sends a \textit{Location Update Request} to the network, and if the MS 
    change the LA, the position database is updated.\\ 
    This procedure is also very useful to know where the device is, and the process to find out
    exactly where the device is called \textbf{paging}. All base stations in LA broadcast a paging 
    message with the ID of the called user. When the mobile terminal replies, the network knows
    where the device is, and can establish a connection.
    \begin{figure}[h]
      \centering
      \includegraphics[width=0.5\textwidth]{img/wireless/mobile paging.png}
      \caption{An overview of the paging process}
      \label{fig:paging}
    \end{figure}
    \begin{subsubsection}{Paging vs Location update}
      The larger are the LA, the less frequent are the location updates, but the more frequent are 
      the paging messages. The smaller are the LA, the more frequent are the location updates, but
      the less frequent are the paging messages. The size of the LA is a trade-off between the
      frequency of the location updates and the frequency of the paging messages.
      This also depends on the mobility of the users, and the arrival frequency of the calls.
    \end{subsubsection}
  \end{subsection}
\end{section}

\begin{section}{Signalling}
  Signalling, in mobile networks, is in charge of establishing, maintaining, and terminating a
  service, assigning resources or modify them.
  In a classic telephone service, the signalling is in charge of routing and setting up circuits for
  phones, while in a mobile(IP) network, the signalling is used to set up media sessions.\\
  The basic service provided by signalling is the Basic Call that is used to set up a telephone
  call. Signalling hash two main components: the user signalling and the network signalling.
  \begin{paragraph}{User signalling}
    The user signalling is used to allow communication between user terminals and the network, and
    in particular to:
    \begin{itemize}
      \item ask for a service
      \item indicate the called party
    \end{itemize}
    The required information about the call status are provided by the network.
  \end{paragraph}
  \begin{paragraph}{Network signalling}
    The network signalling is used to allow communication between switching stations, and in
    particular to:
    \begin{itemize}
      \item route calls among the available paths
      \item allocate resources
      \item management operations
    \end{itemize}
    It is also used for providing supplementary services like special numbers(like 911), calling
    party notifications, mobility management, and so on.
  \end{paragraph}
  \begin{figure}[h]
    \centering
    \includegraphics[width=0.6\textwidth]{img/wireless/signalling.png}
    \caption{An overview of the signalling process for a simple call}
    \label{fig:signalling}
  \end{figure}
  \begin{subsection}{Digital user signalling}
    We are now in a digital era, and the user signalling is now digital. ISDN, which stands for
    Integrated Services Digital Network, is a set of communication standards for simultaneous
    digital transmission of voice, video, data, and other network services over the traditional
    circuits of the public switched telephone network. Digital interfaces works in
    a rather different fashion, because the information is sent in digital messages transmitted over
    L2 frames in the D channel of the ISDN interface. The set of protocols for user signalling
    defines the DSS1 signalling system, which are transported directly into LDAP frames.

    \begin{figure}[h]
      \centering
      \includegraphics[width=0.4\textwidth]{img/wireless/digital user signalling.png}
      \caption{An overview of the digital signalling process}
      \label{fig:digital-signalling}
    \end{figure}

    \begin{figure}[h]
      \centering
      \includegraphics[width=0.5\textwidth]{img/wireless/digital user signalling exhange.png}
      \caption{The exchange of figure \ref{fig:signalling} in digital signalling}
      \label{fig:digital-signalling-2}
    \end{figure}
  \end{subsection}
  
  \begin{subsection}{Network signalling}
    Network signalling is a old standard, but still used today. The,kind of, modern signalling stack
    is called \textbf{SS7}, which stands for Signalling System 7, and is a set of telephony
    signalling protocols used for telephony, and which evolved to support all the services of modern
    fixed and mobile networks.\\
    The signalling is performed out-of-band, meaning that the signalling messages are transmitted
    over a separate network, and not over the same network used for the voice calls, and performed
    via packet switching.\\
    SS7 defines different kind of nodes:
    \begin{itemize}
      \item \textbf{Service Switching Points(SSP)} are the nodes that switch the calls and are
        responsible for the call control. They convert global titles digits(ie. phone numbers) into
        ss7 signalling messages.
      \item \textbf{Service Control Points(SCP)} are the nodes that provide application access to
        the network, and are responsible for the service logic.
      \item \textbf{Signal Transfer Points(STP)} are the nodes that route the messages between the
        SSPs and SCPs, and are responsible for the routing of the messages.
    \end{itemize}

    \begin{figure}[h]
      \centering
      \includegraphics[width=0.5\textwidth]{img/wireless/ss7 architecture.png}
      \caption{An overview of the SS7 architecture}
      \label{fig:ss7}
    \end{figure}

    For the datagram packet switched network, The lower part of the SS7 protocol stack includes:
    \begin{itemize}
      \item Physical layer(MTP-1) defines the physical and electrical characteristics of the
        signalling links of the SS7 network, Signalling links utilise DS–0 channels and carry raw
        signalling data at a rate of 56 kbps or 64 kbps.
      \item Data link layer(MTP-2) provides link-layer functionality, in particular reliability
        and error detection.
      \item Network layer(MTP-3) provides routing functionality, and ensures that messages can be
        delivered between signalling points across the SS7 network regardless of whether they are
        directly connected. It also provides capabilities like node addressing, routing, alternate
        routing, and congestion control.
    \end{itemize}

    \begin{figure}[h]
      \centering
      \includegraphics[width=0.5\textwidth]{img/wireless/ss7 digital call.png}
      \caption{The same call of figure \ref{fig:signalling} by the ss7 point of view}
      \label{fig:ss7-call}
    \end{figure}
    \begin{subsubsection}{Security in ss7}
      SS7 was designed in an era when security concerns were not considered. Every node in the SS7
      is trusted, because no authentication is performed, and the messages are not encrypted. 
      A lot of nasty attack can be performed, like:
      \begin{itemize}
        \item \textbf{Location tracking} by sending a location request to the network, and the
          network will reply with the location of the user.
        \item \textbf{SMS interception} by sending a SMS request to the network, and the network will
          reply with the SMS.
        \item \textbf{Call interception} by sending a call request to the network, and the network
          will reply with the call.
        \item \textbf{Denial of service} by sending a lot of messages to the network, and the network
          will be overloaded.
        \item \textbf{Fraud} by sending a call request to the network, and the network will reply
          with the call, but the call will be charged to the victim.
      \end{itemize}
      All those attacks are still possible today.
      \begin{figure}[h]
        \centering
        \includegraphics[width=0.5\textwidth]{img/wireless/ss7 attacks.png}
        \caption{An overview of the SS7 attack}
        \label{fig:ss7-attack}
      \end{figure}
    \end{subsubsection}
  \end{subsection}
\end{section}

\begin{section}{GSM network}
  The first network architecture of GSM was for circuit switched (CS) services only, and telephone
  service in particular

\end{section}
