\chapter{WLAN technologies and security}
With WLAN, as previously mentioned, we refer to a network that covers a small area. Actually, the most
common WLAN technology is Wi-Fi, which is based on IEEE 802.11 and trademarked.\\
Wifi supports mainly the infrastructure-based mode: which is usually made up of 
\begin{itemize}
  \item Access Points(APs)
  \item Wireless stations(STAs)
\end{itemize}
All the STAs are connected to the APs, which are connected to the wired network.\\
It actually supports ad-hoc mode, which uses direct communication, but it is not used that much.\\
  In infrastructure-based mode, all traffic must go through the AP, having no direct communication 
  possible. This is to solve the hidden terminal problem, where two STAs can't communicate because they
  can't hear each other, and 
\begin{figure}[h]
  \centering
  \includegraphics[width=0.7\textwidth]{img/wireless/wlan communication.png}
  \caption{Schema of a possible wireless communication}
\end{figure}
Furthermore, because it is based on 802.11, it carries over most of the standard:
\begin{itemize}
  \item \textbf{half-duplex communication channel}, at any time only one station can transmit or receive
  \item works on 2.4GHz and 5GHz bands, so multiplexing is available but quite limited
  \item uses \textbf{CSMA/CA} for medium access(carrier sense multiple access with collision avoidance)
\end{itemize}
The spectrum of the 2.4GHz band is divided into 14 channels, but only 3 of them are non-overlapping.
The 5GHz band has more channels, but it is less used because of the shorter range.
\begin{figure}
  \centering
  \includegraphics[width=0.5\textwidth]{img/wireless/80211 channels.png}
  \caption{802.11 channels}
\end{figure}
\begin{section}{WLAN Access}
  Before being able to access the network, an be able to send traffic trough an AP, a 
  mobile station can be:
  \begin{itemize}
    \item \textbf{Unauthenticated and unassociated}: the STA is not authenticated and not associated
    \item \textbf{Authenticated and unassociated}: the STA is authenticated but cannot transmit of receive 
      data
    \item \textbf{Authenticated and associated}: the STA is authenticated and associated
  \end{itemize}
  Data exchange is only possible in the last case.
  \begin{paragraph}{Authentication}
    \label{par:authentication}
    The 802.11 authentication is the first step to be performed. It requires the STA to prove its identity
    to the AP. During this step, no data encryption or security in general is provided.\\
    Usually the authentication is done with a shared key(WEP, WPA, WPA2, WP3), but it can also be 
    done with open system authentication, where the AP accepts any STA that wants to connect.\\
  \end{paragraph}
  \begin{paragraph}{Association}
    The 802.11 association is the second step to be performed. It requires the STA to select the AP to connect to,
    and to exchange information with it to gain access to the network.\\
    Association allows the AP to record each STA so that frames can be sent to the correct destination.
    Association only works in infrastructure mode, and a station can only be associated with one AP at a time.
    It is usually carried out as follows:
    \begin{itemize}
      \item After authentication, the STA sends an association request to the AP
      \item The AP processes the Association Request and decides if a client request should be allowed
        \subitem If the request is accepted it responds with a status code of 0 (successful) and the 
        Association ID (AID)
        \subitem Failed Association Requests include only a status code and the procedure ends
    \end{itemize}
  \end{paragraph}
  The STA must go through the following steps:
  \begin{enumerate}
    \item \textbf{Scanning(Beacon/probe)}: the STA scans the environment to find the APs
    \item \textbf{Association}: the STA selects the AP and sends an association request
    \item \textbf{Authentication}: the AP sends an authentication request to the STA
    \item \textbf{Authentication}: the STA sends an authentication response to the AP
    \item \textbf{Association}: the AP sends an association response to the STA
    \item \textbf{Exchange of data}: the STA can now exchange data with the AP
  \end{enumerate}
  Authentication can be performed with multiple access points of the same network, but association can only
  be performed with one of them.
  \begin{subsection}{Scanning for Access Points}
    To access the network, the mobile station must first associate with an access point.\\
    The \textbf{scanning process} is used to find the APs in the area, and it can be:
    \begin{itemize}
      \item \textbf{Passive scanning}: the STA listens to the beacon frames sent by the APs
      \item \textbf{Active scanning}: the STA sends a probe request to the APs in the area,
        and the APs respond with a probe response.
    \end{itemize}
    The beacon frames contains the informations needed to connect to the AP, such as the SSID, the
    mac address, the supported data rates, \dots, and every access point sends a beacon frame periodically, 
    usually every 100ms, as requested by the standard, even tough it can be disabled.\\
    Those informations are also contained in the probe response, which is sent by the AP in response to the
    probe request.
    \begin{figure}[h]
      \centering
      \includegraphics[width=0.8\textwidth]{img/wireless/wlan scanning.png}
      \caption{The two scanning methods}
    \end{figure}
  \end{subsection}
\end{section}
\begin{section}{Problems with 802.11}
  As already said in the chapter about wireless, the medium is unrealibale and the channel is shared, so
  there are some problems that have to be addressed.
  \begin{subsection}{Hidden terminal problem}
    The hidden terminal problem is a problem that arises when two stations can't hear each other, but they
    can hear the same access point.\\
    This can happen, for example, when two stations are too far from each other, but they are close 
    to the AP, or there's an obstacle between them.\\
    This problem is illustrated in figure \ref{fig:hidden terminal problem}.
    \begin{figure}[h]
      \centering
      \includegraphics[width=0.5\textwidth]{img/wireless/hidden terminal problem.png}
      \caption{Hidden terminal problem}
      \label{fig:hidden terminal problem}
    \end{figure}
    Suppose that Station A is transmitting to Station B, which is also receiving from Station C.\\
    Station C can't hear Station A, so it will transmit, causing a collision at Station B.
  \end{subsection}
  \begin{subsection}{Fading}
    A second scenario that results in undetectable collisions at the receiver is fading.\\
    With fading we refer to the fact that the signal loses strength as it propagates through the medium.\\
    Figure \ref{fig:fading} shows the fading problem. A and C are placed such that their signals 
    are not strong enough to detect each other's transmissions , yet their signals are strong 
    enough to interfere with each other at station B.
    \begin{figure}[h]
      \centering
      \includegraphics[width=0.5\textwidth]{img/wireless/fading.png}
      \caption{Fading}
      \label{fig:fading}
    \end{figure}
  \end{subsection}
  As previously said, 802.11 uses a half-duplex channel, so only one station can transmit at a time.
  \begin{subsection}{CSMA/CA}
    \label{sub:CSMA/CA}
    Since the channel is shared, the 802.11 standard uses CSMA/CA to avoid collisions, which is a 
    random access protocol similar to the CSMA/CD used in Ethernet.\\
    It implements collision avoidance instead of detection for two reasons:
    \begin{itemize}
      \item I requires a full duplex channel, and because the signals are very weak, it can be very
        costly to implement
      \item fading and hidden terminal problems make it difficult to detect collisions
    \end{itemize}
    For those reasons, once a station begins to transmit a frame, it transmits the frame in its 
    entirety, which, if you think about it, would increase the probability of a collision, degrading
    the performance of the network.\\
    Keep also in mind that 802.11 uses link-layer acknowledgements, so if an acknowledgement is not
    received after a certain time, it is retransmitted, and after a certain number of retransmissions,
    the frame is dropped.
    The CSMA/CA protocol is as follows: suppose that a station wants to transmit a frame
    \begin{enumerate}
      \item If initially the station senses the channel idle, it transmits its frame after a short 
        period of time known as the Distributed Inter-frame Space (DIFS)
      \item Otherwise, the station chooses a random backoff value using binary exponential backoff 
         and counts down this value after DIFS when the channel is sensed idle. While the channel 
         is sensed busy, the counter value remains frozen.
      \item When the counter reaches zero (note that this can only occur while the channel is 
        sensed idle), the station transmits the entire frame and then waits for an acknowledgment.
      \item If an acknowledgment is received, the transmitting station knows that its frame has 
        been correctly received at the destination station. If the station has another 
        frame to send, it begins the CSMA/CA protocol at step 2. If the acknowledgment isn't 
        received, the transmitting station reenters the backoff phase in step 2, with the random 
        value chosen from a larger interval.
    \end{enumerate}
    The goal is thus avoiding collisions whenever possible, hoping that by choosing a random
    backoff value, the stations will not choose the same value, and thus will not transmit at the
    same time.\\
    \begin{subsubsection}{RTS and CTS}
      802.11 also includes a optional reservation scheme, which is used to reserve the channel before
      transmitting a frame.\\
      The station that wants to transmit a frame can send a short \textbf{Request to Send(RTS)} 
      control frame and a short \textbf{Clear to Send(CTS)} control frame to \textit{reserve} the
      access to the channel.\\
      When a sender wants to send a DATA frame, it can first send an RTS frame to the AP, indicating 
      the total time required to transmit the DATA frame and the acknowledgment (ACK) frame. When 
      the AP receives the RTS frame , it responds by broadcasting a CTS frame. This CTS frame serves 
      two purposes: It gives the sender explicit permission to send and also instructs the other 
      stations not to send for the reserved duration.\\
      This technique make other station refrain from sending, thus reducing the probability of
      collisions, and solving the hidden terminal problem, at the cost of a higher overhead.\\
      \begin{figure}[h]
        \centering
        \includegraphics[width=0.5\textwidth]{img/wireless/cmsa with RTS.png}
        \caption{Collision avoidance with RTS/CTS}
      \end{figure}

    \end{subsubsection}

  \end{subsection}
\end{section}
\begin{section}{802.11 Features}
  \begin{subsection}{Frame Addressing}
    Although the 802.11 frame shares many similarities with an Ethernet frame, it also contains a 
    number of fields that are specific to its use for wireless links.\\
    The frame format is shown in figure \ref{fig:80211 frame format}.
    \begin{figure}[h]
      \centering
      \includegraphics[width=0.5\textwidth]{img/wireless/802.11 frame.png.png}
      \caption{802.11 frame format}
      \label{fig:80211 frame format}
    \end{figure}
  \end{subsection}
  The numbers above each of the fields in the frame represent the lengths of the fields in bytes.
  The duration field is used to indicate the amount of time required to transmit the frame, and is 
  used for power saving.\\
  After that there are 4 address fields:
  \begin{itemize}
    \item The \textbf{address 1} is the MAC address of the station that is to receive the frame.
    \item The \textbf{address 2} is the MAC address of the station that is transmitting 
      the frame.
    \item The \textbf{address 3} contains the MAC address of the router interface of the subnet.
    \item The \textbf{address 4} is used in ad-hoc mode, and is the MAC address of the 
      destination station.
  \end{itemize}
  \begin{subsection}{Rate adaptation}
    802.11 also supports rate adaptation, which is the ability to change the transmission rate 
    based on the quality of the channel.\\
    This is necessary because the channel conditions can change rapidly, especially with mobile
    stations. In fact, As node moves away from base station, SNR decreases, BER(Bit Error Rate) increases
    If the modulation technique used in the 802.11 protocol operating between the base
    station and the user does not change, the BER will become unacceptably high as the
    SNR decreases, and eventually no transmitted frames will be received correctly.
    Base station and mobile dynamically change transmission rate (physical layer modulation technique) 
    as the mobile station moves, and the SNR varies as a result.
  \end{subsection}
  \begin{subsection}{Power Management}
    \label{sub:power management}
    Power is a precious resource in mobile devices, and thus the 802.11 standard provides 
    power-management capabilities that allow 802.11 nodes to minimize the amount of time that 
    their sense, transmit, and receive functions and other circuitry need to be "on".\\
    A node is able to explicitly alternate between sleep and wake states. A node indicates to the 
    access point that it will be going to sleep by setting the power-management bit in the header 
    of an frame. A timer in the node is then set to wake up the node just before the AP is 
    scheduled to send its beacon frame.\\
    Since the AP knows from the set power-transmission bit that the node is going to sleep, it wont send any
    , and will buffer any frames destined for the sleeping host for later transmission.\\
    A node will wake up just before the AP sends a beacon frame, and quickly enter the fully active 
    state.
    The beacon frames sent out by the AP contain a list of nodes whose frames have been buffered 
    at the AP. If there are no buffered frames for the node, it can go back to sleep. Otherwise, 
    the node can explicitly request that the buffered frames be sent by sending a polling message 
    to the AP.

    \begin{figure}[h]
      \centering
      \includegraphics[width=0.5\textwidth]{img/wireless/80211 power saving.png}
      \caption{Power management in 802.11}
    \end{figure}

  \end{subsection}

\end{section}
\begin{section}{Performance evaluation of 802.11}
  As seen in the previous sections, 802.11 has some features that can be used to improve the
  performance of the network, and is also divided into different version with different 
  throughput: 802.11b(up to 11Mbps), 802.11a(up to 54Mbps), 802.11g(up to 54Mbps), 802.11n(up to 600Mbps),
  \dots\\
  Event tough the throughput is high, the actual throughput is much lower, because of the overhead
  introduced by the protocol, the hidden terminal problem, the collision avoidance, and the fading.\\
  The throughput can be defined as the speed at which useful data is received at the application layer,
  Given the capacity C at the physical layer, and can be calculated as follows:
  \begin{equation}
    T = \frac{\text{(Useful data at the application layer)}}{\text{(time to complete the tranfer)}<C}
  \end{equation}
  which will of course be lower than the capacity C.\\
  Given this definition, we can observe that the throughput is reduced by each layer of the protocol stack,
  because each of them adds control information that is useful to provide the service but is not part of
  the useful data, and that the actual throughput is much lower than the theoretical throughput.\\
  \begin{figure}[h]
    \centering
    \includegraphics[width=0.5\textwidth]{img/wireless/encapsulation overhead.png}
    \caption{The overhead that each packet has to carry}
  \end{figure}
  Furthermore, several factors can reduce the throughput, for example:
  \begin{itemize}
    \item shared physical link with other protocols and transmissions
    \item errors: bit error, congestion, packets drops, you name it
    \item retransmissions: if a packet is lost, it has to be retransmitted
    \item \dots 
  \end{itemize}
  This means that the actual throughput is always less than the protocol efficiency, which is the
  ratio between the useful data and the total data sent per the capacity of the channel.\\

  
  \begin{table}[h]
    \centering
    \begin{tabular}{|c|c|}
      \hline
      Protocol & Overhead \\
      \hline
      TCP & 20 bytes per segment\\
      \hline
      UDP & 8 bytes\\
      \hline
      IP & 20 bytes\\
      \hline
      Ethernet & 38 bytes(8 (trailer + SoF) + 6 (MAC) + 6(MAC) + 2 (Type) + 4 (CRC) + 12 (IPG))\\
      \hline
    \end{tabular}
    \caption{Overhead introduced at each layer per packet}
  \end{table}
  % tabular 2 columns: protocol|overhead
  \begin{subsubsection}{Efficiency of TCP/IPv4/Ethernet}
    The efficiency of the TCP/IP/Ethernet stack can be calculated as follows: we know that the 
    maximum MTU for ethernet is 1500 bytes, and that the maximum payload for an IP packet is 1500-20=1480
    bytes, and that the maximum payload for a TCP segment is 1480-20=1460 bytes.\\
    This overhead is applied to each segment, and because its more likely that the payload at application 
    level is larger than the TCP MTU, there will probably be more segments.\\
    So the efficiency of the TCP/IP/Ethernet stack is:
    \begin{equation}
      \eta_{TCP} = \frac{\text{L7 data}}{\text{Transmitted data}} = \frac{1460}{1460+20+20+20+38} = 94\%
    \end{equation}
    This was in case the channel if full-duplex, but if it is half-duplex, the efficiency is reduced, because
    the ack packets have to be sent in the opposite direction, and the channel can't be used to send data
    in that direction.\\
    The efficiency of the TCP/IP/Ethernet stack in half-duplex is:
    \begin{equation}
      \eta_{TCP} = \frac{\text{L7 data}}{\text{Transmitted data}} = \frac{1460}{(1460+20+20+20)+(20+20+38)} = 90\%
    \end{equation}
    In general, To calculate the total efficiency, it is sufficient to make the product of the
    individual efficiencies.
  \end{subsubsection}
  \begin{subsection}{Theoretical Maximum Throughput}
    The theoretical maximum throughput is a difficult value to calculate, because different parts 
    of the frame are transmitted at different rate: for example, the header is transmitted at a 
    lower rate than the payload(1Mbps).\\
    As such, the theoretical maximum throughput of an application which uses 802.11 is calculated as follows:
    \begin{equation}
      TMT_{APP} = \frac{\beta}{\alpha + \beta}\times TMT_{802.11}(\text{bps})
    \end{equation}
    where $\alpha$ is the total overhead above the MAC layer, and $\beta$ is the application layer
    datagram size.\\
    Computing $TMT_{802.11}$ is a bit more difficult, because some components of the protocol are variable
    but we can approximate it when the scenario is ideal:
    \begin{itemize}
      \item the bit to error rate is 0
      \item there are no losses due to collisions
      \item no packet is lost at the receiving node
      \item PCF mode is not used 
      \item the sending node always has a sufficient number of packets to send
      \item no fragmentation is used at level 2
    \end{itemize}
    but in general we need to consider the ratio between the time in which useful data is sent 
    and the one needed for the overhead.\\
    We can thus consider the TMT as:
    \begin{equation}
      TMT = \frac{\text{MSDU size}}{\text{Delay per MSDU}}
    \end{equation}
    where MSDU is MAC service data unit, which carries the actual data that needs to be transmitted,
    and the delay per MSDU is the time needed to transmit the MSDU, which is overhead time + data time:
    \begin{equation}
      \text{Delay per MSDU} = (T_{DIFS} + T_{SIFS} + T_{BO} + T_{RTS} + T_{CTS} + T_{ACK} + T_{DATA}) \times 10^{-6}s
    \end{equation}
    All those values varies on the mode used:
    \begin{figure}[H]
      \centering
      \includegraphics[width=0.7\textwidth]{img/wireless/delay components.png}
    \end{figure}
  \end{subsection}
\end{section}
\begin{section}{Security issues of WLAN}
  More and more security issues arise when using a wireless medium, and this is also true when using
  802.11.
  \begin{subsection}{Denial of Service}
    Denial of Service is a quite common attack at layer 2, and also quite old, since it has been demonstrated in
    2003, and it is still possible today.
    \begin{boxH}
      Denial of Service is a type of attack that aims to make a network service unavailable to 
      a user, by overwhelming the client with more traffic that it can handle.
    \end{boxH}
    In fact, 802.11 is "more vulnerable" than ethernet, because it is a shared medium, and the attacker
    can easily listen and transmit any frame.\\
    A dos attack can be performed either:
    \begin{itemize}
      \item \textbf{at the physical layer}: by sending a jamming signal, the attacker can prevent the 
        receiver from receiving the frame
      \item \textbf{at the MAC layer}: which mostly rely on the identification problems of 802.11
    \end{itemize}
    With identification problems we refer to the fact that nodes are identified by their MAC address,
    which can be easily spoofed, because there's no authentication at the MAC layer.\\
    Based on this simple fact, there are several attacks that can be performed.
    \begin{subsubsection}{Disassociation attack}
      Before going over this kind of attack, a quick reminder that a client can authenticate with
      multiple APs, but can only be associated with one of them.\\
      There's also another kind of frame, the disassociation frame, which is used to terminate the
      connection between a client and an AP, which is not authenticated, like the association one.\\
      This meas that an attack can, at any point of the connection, send a disassociation frame to the
      AP while spoofing the MAC address of the client, and the AP will terminate the connection with the
      client.\\
      By repeating this process, the attacker can prevent the client from connecting to the network, and
      thus from receiving any data, performing a denial of service attack.
      \begin{figure}[h]
        \centering
        \includegraphics[width=0.6\textwidth]{img/wireless/disassociation attack.png}
        \caption{Example of a disassociation attack}
      \end{figure}

    \end{subsubsection}
    \begin{subsubsection}{Deauthentication attack}
      The general principle of 802.11 authentication has already been explained in paragraph
      \ref{par:authentication}, but it is worth mentioning that the deauthentication frame is used to
      terminate the authentication process.\\
      As such, it is possible to perform a deauthentication attack, in the same fashion as the
      disassociation attack, by sending a deauthentication frame to the AP while spoofing the MAC address
      of the client.\\
      Doing so, the client must re-authenticate with the AP, and if the attack is repeated, the client
      will be unable to connect to the network, and thus to receive any data, performing a denial of
      service attack.\\
      This kind of attack can be executed on individual clients, or on the whole network, by spoofing
      the AP address, and sending the deauthentication frame to all the clients.

      \begin{figure}[h]
        \centering
        \includegraphics[width=0.6\textwidth]{img/wireless/deauthentication attack.png}
        \caption{Example of a deauthentication attack}
      \end{figure}

    \end{subsubsection}
    \begin{subsubsection}{Protection against deauthentication attacks}
      In general, it is better to implement a protection against deauthentication attacks, because 
      after a deauthentication attack, the client will need 2RTTs to resume the communication,
      while after a disassociation attack, the client will need only 1RTT.\\
      A simple protection against deauthentication attacks that as been proposed is based on the 
      observation that legitimate nodes do not deauthenticate themselves and then send data.
      As such, an AP can delay the honoring of a deauthentication frame for a short period of time,
      and if any frames are sent by the client, the deauthentication frame is ignored.\\
      This solution is quite good because it requires no modification to the protocol and its 
      backward compatible.\\
      Another solution is to use the SNR of the signal as a "signature" of the client, and if the
      SNR doesn't match the signature, the deauthentication frame is ignored.
    \end{subsubsection}
    \begin{subsubsection}{Attacks on CSMA-CD}
      The general principle of CSMA-CD has already been explained in paragraph \ref{sub:CSMA/CA},
      but it is worth mentioning that it is possible to perform a denial of service attack by
      sending a signal before every SIFS slot, which will prevent the other nodes from transmitting
      the frame, because they will sense the channel as busy.\\
      So just by sending 50000 frames per second, which is roughly one tenth of the capacity of the
      channel, the attacker can shut down the channel.\\
      The attack that has been just explained only works if RTS/CTS is not used, but its also possible
      to perform an attack even in the opposite case. In fact, in a RTS request it is possible to 
      specify the reservation time, with a 2 byte field, and by setting it to a large value,
      the attacker can prevent the other nodes from transmitting. 
      \begin{figure}[h]
        \centering
        \includegraphics[width=0.6\textwidth]{img/wireless/RTS attack.png}
        \caption{Example of a CSMA-CD/RTS attack, the attacker sets a large value in the RTS request}
      \end{figure}
      To prevent this kind of attack it is possible to set a reasonable limit on the reservation time,
      because all legitimate traffic use relatively small values. 
    \end{subsubsection}
    \begin{subsubsection}{Attack on power saving}
      As seen in in subsection \ref{sub:power management}, 802.11 has a power saving mode, which allows
      the client to go to sleep, and wake up only when it is needed.\\
      To signal that a station is going to sleep, it sends a Null data frame with the power management
      bit set.\\
      Multiple attacks can be performed on this mechanism:
      \begin{itemize}
        \item An attacker can spoof on behalf of AP the TIM message, which is used to signal the clients
          that there are packets waiting for them, making the clients think that there are no packets
          waiting for them, and thus go to sleep
        \item An attacker forge management sync packets, causing them to fall out of sync with the AP
        \item An attacker spoof on behalf of the client, making the AP send data while the client is 
          sleeping
      \end{itemize}
      Let's now describe the latter scenario: of course an attacker can impersonate the victim STA,
      because he just needs to know the MAC addresses of the AP and the victim. The attacker can 
      keep injecting \textbf{false power management frames} to the AP, which will make the AP think
      that the victim is going to sleep, and thus it will buffer the packets.\\
      The AP sends a Beacon frame with the TIM (Traffic Indication Map) to signal the victim to 
      wake-up, but the victim will ignore those, because it's not in power management mode.\\
      \begin{figure}[h]
        \centering
        \includegraphics[width=0.7\textwidth]{img/wireless/power saving attack.png}
        \caption{Example of a power saving attack}
      \end{figure}

    \end{subsubsection}

  \end{subsection}
  Although all those issues were published in 2003, the threats are still present today, making DoS 
  an deauthentication attacks still possible today.

\end{section}
\begin{section}{WLAN security}
    As per the previous section, we also learned that:
    \begin{itemize}
      \item MAC authentication is useless, because is sent in broadcast, so can be sniffed and replayed
      \item static IP addresses are useless, because can be easily sniffed from the packets
      \item SSID hiding is useless, because even if we can suppress beacons frames, SSID are sent 
        in clear text in probe request/response and association response
      \item Smart antenna placement and hiding is useless, because the attacker can have more resources 
        and this also impact the final users
    \end{itemize}
  \begin{subsection}{WEP}
    \begin{boxH}
      WEP stands for Wired Equivalent Privacy, and is a security protocol used in wireless networks.
    \end{boxH}
    It was first introduced in 1998, so it's a quite old standard, and was intended to protect 
    wireless confidentiality comparable with that of a traditional wired network and intended to 
    achieve the following security goals:
    \begin{itemize}
      \item \textbf{Confidentiality} via encryption 
      \item \textbf{Access control} via shared password authentication
      \item \textbf{Data integrity} via checksums
    \end{itemize}
    Actually WEP was never meant to achieve strong security, but it's better than nothing, even if 
    it has some serious flaws, which make it fail to achieve the security goals, which were:
    \begin{itemize}
      \item To be reasonably strong, by changing frequently the \textit{IV} by to avoid statistical attacks
      \item To be self-synchronizing, to be able to decrypt the messages, which is provided by the IV
      \item To be efficient, by using a stream cipher
      \item To be IP free
      \item To be exportable, which is a requirement for the US government
    \end{itemize}
    \begin{subsubsection}{WEP access control}
      WEP implements two different authentication methods:
      \begin{itemize}
        \item \textbf{Open system authentication}: which is the default one, and is used when the 
          client wants to connect to the network
        \item \textbf{Shared key authentication}: which is used when the client wants to connect to the 
          network, and is based on a shared key
      \end{itemize}
      The shared key authentication is based on a challenge-response mechanism, where the AP sends a
      challenge to the client, which has to respond with the correct response, which is based on the
      shared key.\\
      Once authenticated, the STA can send an association request and the AP will respond with an 
      association response. If authentication fails, no association is possible.
      The challenge is based on the RC4 algorithm, which is a stream cipher which is known to be not 
      the most secure nowadays, but at the time was considered secure enough.\\
      % TODO: ADD clear image
      Furthermore, counterintuitively, the shared key authentication is less secure than the open system
      authentication, because a bruteforce attack can be mounted on the challenge frame, because both 
      the cyphertext and the algorithm are known.\\
      If privacy is a primary concern, it is more advisable to use Open System authentication for WEP
      authentication.

      \begin{figure}[h]
        \centering
        \includegraphics[width=0.5\textwidth]{img/wireless/WEP security.png}
        \caption{General overview of WEP security}
      \end{figure}
    \end{subsubsection}
    \begin{subsubsection}{WEP integrity}
      As for what concerns the integrity of the data, WEP uses a checksum, which is a CRC32(Cyclic
      Redundancy Check) checksum of 32 bits, which is appended to the frame.
    \end{subsubsection}
    \begin{subsubsection}{WEP Key Management}
      Two kind of keys are used in WEP:
      \begin{itemize}
        \item \textbf{Default key}, also called \textit{shared key}
        \item \textbf{Key mapping keys}, also called \textit{individual keys}
      \end{itemize}
      but in practice, only the default key is used, because the key mapping keys are not used in the
      standard. This has some serious implication, because each STA can decrypt the frames sent by
      another STA, because they all use the same key.\\
      Furthermore, the default key has to be changed if a member leaves the group, to make it's access
      impossible, but this is also impossible, because the key has to be changed on every device 
      simultaneously, which is quite difficult.\\
      For those reasons, WEP supports multiple default keys, with only the active key used for encryption, 
      and any of the keys can be used for decryption.\\
      The message header contains a key ID that allows the receiver to find out which key should be used
      to decrypt the message.

      \begin{figure}[h]
        \centering
        \includegraphics[width=0.5\textwidth]{img/wireless/WEP key management.png}
        \caption{WEP key management}
      \end{figure}

    \end{subsubsection}

  \end{subsection}

\end{section}


